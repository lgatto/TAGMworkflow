\documentclass[]{article}
\usepackage{lmodern}
\usepackage{amssymb,amsmath}
\usepackage{ifxetex,ifluatex}
\usepackage{fixltx2e} % provides \textsubscript
\ifnum 0\ifxetex 1\fi\ifluatex 1\fi=0 % if pdftex
  \usepackage[T1]{fontenc}
  \usepackage[utf8]{inputenc}
\else % if luatex or xelatex
  \ifxetex
    \usepackage{mathspec}
  \else
    \usepackage{fontspec}
  \fi
  \defaultfontfeatures{Ligatures=TeX,Scale=MatchLowercase}
\fi
% use upquote if available, for straight quotes in verbatim environments
\IfFileExists{upquote.sty}{\usepackage{upquote}}{}
% use microtype if available
\IfFileExists{microtype.sty}{%
\usepackage{microtype}
\UseMicrotypeSet[protrusion]{basicmath} % disable protrusion for tt fonts
}{}
\usepackage[margin=1in]{geometry}
\usepackage{hyperref}
\hypersetup{unicode=true,
            pdftitle={A Bioconductor workflow for the Bayesian Analysis of Spatial proteomics},
            pdfauthor={Oliver M. Crook, Lisa Breckels, Kathryn S. Lilley, Paul D.W. Kirk, Laurent Gatto},
            pdfborder={0 0 0},
            breaklinks=true}
\urlstyle{same}  % don't use monospace font for urls
\usepackage{color}
\usepackage{fancyvrb}
\newcommand{\VerbBar}{|}
\newcommand{\VERB}{\Verb[commandchars=\\\{\}]}
\DefineVerbatimEnvironment{Highlighting}{Verbatim}{commandchars=\\\{\}}
% Add ',fontsize=\small' for more characters per line
\usepackage{framed}
\definecolor{shadecolor}{RGB}{248,248,248}
\newenvironment{Shaded}{\begin{snugshade}}{\end{snugshade}}
\newcommand{\KeywordTok}[1]{\textcolor[rgb]{0.13,0.29,0.53}{\textbf{{#1}}}}
\newcommand{\DataTypeTok}[1]{\textcolor[rgb]{0.13,0.29,0.53}{{#1}}}
\newcommand{\DecValTok}[1]{\textcolor[rgb]{0.00,0.00,0.81}{{#1}}}
\newcommand{\BaseNTok}[1]{\textcolor[rgb]{0.00,0.00,0.81}{{#1}}}
\newcommand{\FloatTok}[1]{\textcolor[rgb]{0.00,0.00,0.81}{{#1}}}
\newcommand{\ConstantTok}[1]{\textcolor[rgb]{0.00,0.00,0.00}{{#1}}}
\newcommand{\CharTok}[1]{\textcolor[rgb]{0.31,0.60,0.02}{{#1}}}
\newcommand{\SpecialCharTok}[1]{\textcolor[rgb]{0.00,0.00,0.00}{{#1}}}
\newcommand{\StringTok}[1]{\textcolor[rgb]{0.31,0.60,0.02}{{#1}}}
\newcommand{\VerbatimStringTok}[1]{\textcolor[rgb]{0.31,0.60,0.02}{{#1}}}
\newcommand{\SpecialStringTok}[1]{\textcolor[rgb]{0.31,0.60,0.02}{{#1}}}
\newcommand{\ImportTok}[1]{{#1}}
\newcommand{\CommentTok}[1]{\textcolor[rgb]{0.56,0.35,0.01}{\textit{{#1}}}}
\newcommand{\DocumentationTok}[1]{\textcolor[rgb]{0.56,0.35,0.01}{\textbf{\textit{{#1}}}}}
\newcommand{\AnnotationTok}[1]{\textcolor[rgb]{0.56,0.35,0.01}{\textbf{\textit{{#1}}}}}
\newcommand{\CommentVarTok}[1]{\textcolor[rgb]{0.56,0.35,0.01}{\textbf{\textit{{#1}}}}}
\newcommand{\OtherTok}[1]{\textcolor[rgb]{0.56,0.35,0.01}{{#1}}}
\newcommand{\FunctionTok}[1]{\textcolor[rgb]{0.00,0.00,0.00}{{#1}}}
\newcommand{\VariableTok}[1]{\textcolor[rgb]{0.00,0.00,0.00}{{#1}}}
\newcommand{\ControlFlowTok}[1]{\textcolor[rgb]{0.13,0.29,0.53}{\textbf{{#1}}}}
\newcommand{\OperatorTok}[1]{\textcolor[rgb]{0.81,0.36,0.00}{\textbf{{#1}}}}
\newcommand{\BuiltInTok}[1]{{#1}}
\newcommand{\ExtensionTok}[1]{{#1}}
\newcommand{\PreprocessorTok}[1]{\textcolor[rgb]{0.56,0.35,0.01}{\textit{{#1}}}}
\newcommand{\AttributeTok}[1]{\textcolor[rgb]{0.77,0.63,0.00}{{#1}}}
\newcommand{\RegionMarkerTok}[1]{{#1}}
\newcommand{\InformationTok}[1]{\textcolor[rgb]{0.56,0.35,0.01}{\textbf{\textit{{#1}}}}}
\newcommand{\WarningTok}[1]{\textcolor[rgb]{0.56,0.35,0.01}{\textbf{\textit{{#1}}}}}
\newcommand{\AlertTok}[1]{\textcolor[rgb]{0.94,0.16,0.16}{{#1}}}
\newcommand{\ErrorTok}[1]{\textcolor[rgb]{0.64,0.00,0.00}{\textbf{{#1}}}}
\newcommand{\NormalTok}[1]{{#1}}
\usepackage{longtable,booktabs}
\usepackage{graphicx,grffile}
\makeatletter
\def\maxwidth{\ifdim\Gin@nat@width>\linewidth\linewidth\else\Gin@nat@width\fi}
\def\maxheight{\ifdim\Gin@nat@height>\textheight\textheight\else\Gin@nat@height\fi}
\makeatother
% Scale images if necessary, so that they will not overflow the page
% margins by default, and it is still possible to overwrite the defaults
% using explicit options in \includegraphics[width, height, ...]{}
\setkeys{Gin}{width=\maxwidth,height=\maxheight,keepaspectratio}
\IfFileExists{parskip.sty}{%
\usepackage{parskip}
}{% else
\setlength{\parindent}{0pt}
\setlength{\parskip}{6pt plus 2pt minus 1pt}
}
\setlength{\emergencystretch}{3em}  % prevent overfull lines
\providecommand{\tightlist}{%
  \setlength{\itemsep}{0pt}\setlength{\parskip}{0pt}}
\setcounter{secnumdepth}{5}
% Redefines (sub)paragraphs to behave more like sections
\ifx\paragraph\undefined\else
\let\oldparagraph\paragraph
\renewcommand{\paragraph}[1]{\oldparagraph{#1}\mbox{}}
\fi
\ifx\subparagraph\undefined\else
\let\oldsubparagraph\subparagraph
\renewcommand{\subparagraph}[1]{\oldsubparagraph{#1}\mbox{}}
\fi

%%% Use protect on footnotes to avoid problems with footnotes in titles
\let\rmarkdownfootnote\footnote%
\def\footnote{\protect\rmarkdownfootnote}

%%% Change title format to be more compact
\usepackage{titling}

% Create subtitle command for use in maketitle
\newcommand{\subtitle}[1]{
  \posttitle{
    \begin{center}\large#1\end{center}
    }
}

\setlength{\droptitle}{-2em}

  \title{A Bioconductor workflow for the Bayesian Analysis of Spatial proteomics}
    \pretitle{\vspace{\droptitle}\centering\huge}
  \posttitle{\par}
    \author{Oliver M. Crook, Lisa Breckels, Kathryn S. Lilley, Paul D.W. Kirk,
Laurent Gatto}
    \preauthor{\centering\large\emph}
  \postauthor{\par}
    \date{}
    \predate{}\postdate{}
  

\begin{document}
\maketitle

\newcommand{\diag}{\mathop{\mathrm{diag}}}

\section*{Abstract}\label{abstract}
\addcontentsline{toc}{section}{Abstract}

A determinant step in deciphering a proteins function is finding out its
subcellular localisation. Due to improved multiplexing capabilities of
high-throughput mass-spectrometry, the field of spatial proteomics has
become increasingly popular because of its ability to now systematically
localise thousands of proteins per experiment. In parallel with these
experimental advances, there has has also been improvements on the
analysis of spatial proteomics data. In this workflow, we demonstate
using \texttt{pRoloc} for the Bayesian analysis of spatial proteomics
data. We detail the software infrastructure and then provide
step-by-step guidance of the analysis, including setting up a pipeline,
assessing convergence, and interpretting downstream results. In several
places we provide additional details on Bayesian analysis to provide
users with a holisitic view of Bayesian analysis for spatial proteomics
data.

\section{Introduction}\label{introduction}

Quantifying uncertainty in the spatial subcellular distribution of
proteins allows for novel insight into protein function (Crook et al.
2018). Many proteins live in a single location within the cell, however
there are those that reside in mutiple locations and those that
dynamically relocalise (Thul et al. 2017). These phenomena lead to
variability and uncertainty in robustly assigning proteins a unique
localisation. Functional comparmentalisation of proteins allows the cell
to control biomolecular pathways and biochemical processes within the
cell. Therefore, proteins with multiple localisation may have mutiple
functional roles (Jeffery 2009). Machine learning algorithms that fail
to quantify uncertainty are unable to draw deeper insight into
understanding cell biology from mass-spectrometry (MS) based spatial
proteomics experiments. Henceforth, quantifying uncertainty allows us to
make rigorous assessments of protein subcellular localisation and
multi-localisation.

For proteins to carry out their functional role they must be localised
to the correct sub-cellular compartment, ensuring the biochemical
conditions for desired molecular interactions are met (Gibson 2009).
Many pathologies, including cancer and obesity are characterised by
protein mis-localisations (Olkkonen and Ikonen 2006, Laurila and Vihinen
(2009), Luheshi, Crowther, and Dobson (2008), De Matteis and Luini
(2011), Cody, Iampietro, and Lécuyer (2013), Kau, Way, and Silver
(2004), Rodriguez, Au, and Henderson (2004), Latorre et al. (2005), Shin
et al. (2013), Siljee et al. (2018)). High-throughput spatial proteomics
technologies have seen rapid improvement over the last decade and now a
single experiment can provide spatial information on thousands of
proteins at once (Dunkley et al. 2006, Foster et al. (2006),
Christoforou et al. (2016), Geladaki et al. (2019)). As a result of
these spatial protomics technolgies many biological systems have been
characterised (Dunkley et al. 2006, Tan et al. (2009), Hall et al.
(2009), Breckels et al. (2013), Christoforou et al. (2016), Thul et al.
(2017)). The popularity of such methods is now evident with many new
studies in recent years (Christoforou et al. 2016, Beltran, Mathias, and
Cristea (2016), Jadot et al. (2017), Itzhak et al. (2017), Mendes et al.
(2017), Hirst et al. (2018), Davies et al. (2018), Orre et al. (2019),
Nightingale et al. (2019)).

Bayesian approaches to machine learning and statistical analysis can
provide more insight into the data, since uncertainty quantification
arises as a consequence of a generative model for the data (Gelman et
al. 1995). In a Bayesian framework, a model with paramters for the data
is proposed, along with a statement about our prior beliefs of the model
paramters. Bayes' theorem tells us how to update the prior distribution
of the parameters to obtain the posterior distribution of the parameters
after observing the data. It is the posterior distribution which
quantifies the uncertainty in the parameters and quantities of interest
derived from the data. This contrasts from a maximum-likelihood approach
where we obtain only a point estimate of the parameters.

Adopting a Bayesian framework for data analysis, though of much interest
to experimentalists, can be challenging. Once we have obtained a
probabilistic model, complex algorithms are used to obtain the posterior
distribution upon observation of the data. These algorithms can have
tuning parameters and many settings, hindering their practical use for
those not versed in Bayesian methodology. Even once the algorithms have
been correctly set-up, assessments of convergence and guidance on how to
intepret the results are often sparse. This workflow presents a Bayesian
analysis of spatial proteomics to elucidate the process to any
practioners. We hope that it goes beyond simply the methods, data
structures and biology presented here, but provides a template for
others to design tools using Bayesian methodology for the biological
community.

Our model for the data is the t-augmented Gaussian mixture (TAGM) model
proposed in (Crook et al. 2018). Crook et al. (2018) provide a detailed
description of the model, rigorous comparisons and testing on many
spatial proteomics datasets and a case study on a hyperLOPIT experiment
on mouse pluripotent stem cells (Christoforou et al. 2016; Mulvey et al.
2017). Revisiting these details is not the purpose of this computational
protocol, rather we present how to correctly use the software and
provide step by step guidance for interpreting the results.

In brief, the TAGM model posits that each annotated sub-cellular niche
can be described by a Gaussian distribution. Thus the full complement of
proteins within the cell is captured as a mixture of Gaussians. The
highly dynamic nature of the cell means that many proteins are not well
captured by any of these multivariate Gaussian distributions, and thus
the model also includes an outlier component, mathematically described
as multivariate student's t distribution. The heavy tails of the t
distribution allow it to better capture dispersed proteins.

To perform inference in the TAGM model there are two approaches. The
first, which we refer to as TAGM MAP, allows us to obtain \emph{maximum
a posteriori} estimates of posterior localisation probabilities; that
is, the modal posterior probability that a protein localises to that
class. This approach uses the expectation-maximisation (EM) algorithm to
perform inference (Dempster, Laird, and Rubin 1977). Whilst this is a
interpretable summary of the TAGM model, it only provides point
estimates. For a richer analysis, we present a Markov-chain Monte-Carlo
(MCMC) method to perform fully Bayesian inference in our model, allowing
us to obtain full posterior localisation distributions. This method is
refered to as TAGM MCMC throughout the text.

This workflow begins with a brief review of some of the basic features
of mass spectrometry-based spatial proteomics data, including the
state-of-the-art computational infrastructure and bespoke software
suite. We then present each method in turn, detailing how to obtain high
quality results. We provide an extended dicussion of the TAGM MCMC
method to highlight some of the challenges that may arise when applying
this method. This includes how to assess convergence of MCMC methods, as
well as methods for manipulating the output. We then take the processed
output and explain how to intepret the results, as well as providing
some tools for visualisation. We conclude with some remarks and
directions for the future.

\section{Getting started and
infrastructure}\label{getting-started-and-infrastructure}

In this workflow, we are currently using version 1.23.2 of
\texttt{pRoloc} (Gatto et al. 2014). The pacakge \texttt{pRoloc}
contains algorithms and methods for analysing spatial proteomics data,
building on the \texttt{MSnSet} structure provided in \texttt{MSnbase}.
The \texttt{pRolocdata} package provides many annotated datasets from a
variety of species and experimental procedures. The following code
chunks install and load the suite of packages require for the analysis.

\begin{Shaded}
\begin{Highlighting}[]
\NormalTok{if (!}\KeywordTok{require}\NormalTok{(}\StringTok{"BiocManager"}\NormalTok{))}
    \KeywordTok{install.package}\NormalTok{(}\StringTok{"BiocManager"}\NormalTok{)}
\NormalTok{BiocManager::}\KeywordTok{install}\NormalTok{(}\KeywordTok{c}\NormalTok{(}\StringTok{"pRoloc"}\NormalTok{, }\StringTok{"pRolocdata"}\NormalTok{))}
\end{Highlighting}
\end{Shaded}

\begin{Shaded}
\begin{Highlighting}[]
\KeywordTok{library}\NormalTok{(}\StringTok{"pRoloc"}\NormalTok{)}
\end{Highlighting}
\end{Shaded}

\begin{verbatim}
## 
## This is pRoloc version 1.23.2 
##   Visit https://lgatto.github.io/pRoloc/ to get started.
\end{verbatim}

\begin{Shaded}
\begin{Highlighting}[]
\KeywordTok{library}\NormalTok{(}\StringTok{"pRolocdata"}\NormalTok{)}
\end{Highlighting}
\end{Shaded}

\begin{verbatim}
## 
## This is pRolocdata version 1.20.0.
## Use 'pRolocdata()' to list available data sets.
\end{verbatim}

We assume that we have a MS-based spatial proteomics dataset contained
in a \texttt{MSnSet} structure. For information on how to import data,
perform basic data processing, quality control, supervised machine
learning and transfer learning we refer the reader to (Breckels, Mulvey,
et al. 2016). Here, we start by loading a spatial proteomics dataset on
mouse E14TG2a embryonic stem cells (Breckels, Holden, et al. 2016). The
LOPIT protocol (Dunkley et al. 2004; Dunkley et al. 2006) was used and
the normalised intensity of proteins from eight iTRAQ 8-plex labelled
fraction are provided. The methods provided here are independent of
labelling procedure, fractionation process or workflow. Examples of
valid experimental protocols are LOPIT (Dunkley et al. 2004), hyperLOPIT
(Christoforou et al. 2016; Mulvey et al. 2017), label-free methods such
as PCP (Foster et al. 2006), and when fractionation is perform by
differential centrifugation (Itzhak et al. 2016; Geladaki et al. 2019).

In the code chunk below, we load the aforementioned dataset. The
printout demonstrates that this experiment quantified 2031 proteins over
8 fractions.

\begin{Shaded}
\begin{Highlighting}[]
\KeywordTok{data}\NormalTok{(}\StringTok{"E14TG2aR"}\NormalTok{) }\CommentTok{# load experimental data}
\NormalTok{E14TG2aR}
\end{Highlighting}
\end{Shaded}

\begin{verbatim}
## MSnSet (storageMode: lockedEnvironment)
## assayData: 2031 features, 8 samples 
##   element names: exprs 
## protocolData: none
## phenoData
##   sampleNames: n113 n114 ... n121 (8 total)
##   varLabels: Fraction.information
##   varMetadata: labelDescription
## featureData
##   featureNames: Q62261 Q9JHU4 ... Q9EQ93 (2031 total)
##   fvarLabels: Uniprot.ID UniprotName ... markers (8 total)
##   fvarMetadata: labelDescription
## experimentData: use 'experimentData(object)'
## Annotation:  
## - - - Processing information - - -
## Loaded on Thu Jul 16 15:02:29 2015. 
## Normalised to sum of intensities. 
## Added markers from  'mrk' marker vector. Thu Jul 16 15:02:29 2015 
##  MSnbase version: 1.17.12
\end{verbatim}

On figure \ref{fig:e14pca1} (left), we can visualise the mouse stem cell
dataset use the \texttt{plot2D} function. We observe that some of the
organelle classes overlap and this is a typical feature of biological
datasets. Thus, it is vital to perform uncertainty quantification when
analysising biological data.

\begin{Shaded}
\begin{Highlighting}[]
\KeywordTok{plot2D}\NormalTok{(E14TG2aR)}
\KeywordTok{addLegend}\NormalTok{(E14TG2aR, }\DataTypeTok{where =} \StringTok{"topleft"}\NormalTok{, }\DataTypeTok{cex =} \FloatTok{0.6}\NormalTok{)}
\end{Highlighting}
\end{Shaded}

\begin{figure}[htbp]
\centering
\includegraphics{TAGMworkflow_files/figure-latex/e14pca1-1.pdf}
\caption{\label{fig:e14pca1}First two principal components of mouse stem
cell data.}
\end{figure}

\newpage

\section{\texorpdfstring{Methods: \emph{TAGM
MAP}}{Methods: TAGM MAP}}\label{methods-tagm-map}

\subsection{Introduction to TAGM MAP}\label{introduction-to-tagm-map}

We can perform \emph{maximum a posteriori} (MAP) estimation to perform
Bayesian inference in our model. The \emph{maximum a posteriori}
estimate equals the mode of the posterior distribution and can be used
to provide a point estimate summary of the posterior localisation
probabilities. It does not provide samples from the posterior
distribution, however an extended version of the
expectation-maximisation (EM) algorithm can be used in our case,
allowing fast inference. The EM algorithm is an algorithm that iterates
between an expectation step and a maximisation step. This allows us to
find parameters which maximise the logarithm of the posterior, in the
presence of latent (unobserved) variables. The EM algorithm is
guaranteed to converge to a local mode. The code chunk below executes
the \texttt{tagmMapTrain} function for a default of 100 iterations. We
use the default priors for simplicity and convenience, however they can
be changed, which we explain in a later section. The output is an object
of class \texttt{MAPParams}, that captures the details of the TAGM MAP
model.

\begin{Shaded}
\begin{Highlighting}[]
\KeywordTok{set.seed}\NormalTok{(}\DecValTok{2}\NormalTok{)}
\NormalTok{mappars <-}\StringTok{ }\KeywordTok{tagmMapTrain}\NormalTok{(E14TG2aR)}
\end{Highlighting}
\end{Shaded}

\begin{verbatim}
## co-linearity detected; a small multiple of
##               the identity was added to the covariance
\end{verbatim}

\begin{Shaded}
\begin{Highlighting}[]
\NormalTok{mappars}
\end{Highlighting}
\end{Shaded}

\begin{verbatim}
## Object of class "MAPParams"
##  Method: MAP
\end{verbatim}

\subsubsection*{Aside: co-linearity}\label{aside-co-linearity}
\addcontentsline{toc}{subsubsection}{Aside: co-linearity}

The previous code chunk outputs a message concerning data co-linearity.
This is because the covariance matrix of the data has become
ill-conditioned and as a result the inversion of this matrix becomes
unstable with floating point arithmetic. This can lead to the failure of
standard matrix algorithms upon which our method depends. It is standard
practice to add a small ridge; that is, a small multiple of the identity
to stablise this matrix. The printed message is a statement that this
operation has been performed for these data.

\subsection{Model visualisation}\label{model-visualisation}

The results of the modelling can be visualised with the
\texttt{plotEllipse} function on figure \ref{fig:e14ellipse}. The outer
ellipse contains 99\% of the total probability whilst the middle and
inner ellipses contain 95\% and 90\% of the probability respectively.
The centres of the clusters are represented by black circumpunct
(circled dot). We can also plot the model in other principal components.
The code chunk below plots the probability ellipses along the first and
second, as well as the fourth principal component. The user can change
the components visualised by altering the \texttt{dims} argument.

\begin{Shaded}
\begin{Highlighting}[]
\KeywordTok{par}\NormalTok{(}\DataTypeTok{mfrow =} \KeywordTok{c}\NormalTok{(}\DecValTok{1}\NormalTok{, }\DecValTok{2}\NormalTok{))}
\KeywordTok{plotEllipse}\NormalTok{(E14TG2aR, mappars)}
\KeywordTok{plotEllipse}\NormalTok{(E14TG2aR, mappars, }\DataTypeTok{dims =} \KeywordTok{c}\NormalTok{(}\DecValTok{1}\NormalTok{, }\DecValTok{4}\NormalTok{))}
\end{Highlighting}
\end{Shaded}

\begin{figure}[htbp]
\centering
\includegraphics{TAGMworkflow_files/figure-latex/e14ellipse-1.pdf}
\caption{\label{fig:e14ellipse}PCA plot with probability ellipses along PC 1
and 2 (left) and PC 1 and 4 (right)}
\end{figure}

\subsection{The expectation-maximisation
algorithm}\label{the-expectation-maximisation-algorithm}

The EM algorithm is iterative; that is, the algorithm iterates between
an expectation step and a maximisation step until the value of the
log-posterior does not change (Dempster, Laird, and Rubin 1977). This
fact can be used to assess the convergence of the EM algoritm. The value
of the log-posterior at each iteration can be accessed with the
\texttt{logPosteriors} function on the \texttt{MAPParams} object. The
code chuck below plots the log posterior at each iteration and we see on
figure \ref{fig:mapconverge} the algorithm rapidly plateaus and so we
have acheived convergence. If convergence has not been reached during
this time, we suggest to increase the number of iterations by changing
the parameter \texttt{numIter} in the \texttt{tagmMapTrain} method. In
practice, it is not unexpected to observe small fluctations due to
numerical errors and this should not concern users.

\begin{Shaded}
\begin{Highlighting}[]
\KeywordTok{plot}\NormalTok{(}\KeywordTok{logPosteriors}\NormalTok{(mappars), }\DataTypeTok{type =} \StringTok{"b"}\NormalTok{, }\DataTypeTok{col =} \StringTok{"blue"}\NormalTok{,}
     \DataTypeTok{cex =} \FloatTok{0.3}\NormalTok{, }\DataTypeTok{ylab =} \StringTok{"log-posterior"}\NormalTok{, }\DataTypeTok{xlab =} \StringTok{"iteration"}\NormalTok{)}
\end{Highlighting}
\end{Shaded}

\begin{figure}[htbp]
\centering
\includegraphics{TAGMworkflow_files/figure-latex/mapconverge-1.pdf}
\caption{\label{fig:mapconverge}Log-posterior at each iteration of the EM
algorithm demonstrating convergence.}
\end{figure}

The code chuck below uses the \texttt{mappars} object generated above,
along with the \texttt{E14RG2aR} dataset, to classify the proteins of
unknown localisation using \texttt{tagmPredict} function. The results of
running \texttt{tagmPredict} are appended to the \texttt{fData} columns
of the \texttt{MSnSet}.

\begin{Shaded}
\begin{Highlighting}[]
\NormalTok{E14TG2aR <-}\StringTok{ }\KeywordTok{tagmPredict}\NormalTok{(E14TG2aR, mappars) }\CommentTok{# Predict protein localisation}
\end{Highlighting}
\end{Shaded}

The new feature variables that are generated are:

\begin{itemize}
\tightlist
\item
  \texttt{tagm.map.allocation}: the TAGM MAP predictions for the most
  probable protein sub-cellular allocation.
\end{itemize}

\begin{Shaded}
\begin{Highlighting}[]
\KeywordTok{table}\NormalTok{(}\KeywordTok{fData}\NormalTok{(E14TG2aR)$tagm.map.allocation)}
\end{Highlighting}
\end{Shaded}

\begin{verbatim}
## 
##          40S Ribosome          60S Ribosome               Cytosol 
##                    34                    85                   328 
## Endoplasmic reticulum              Lysosome         Mitochondrion 
##                   284                   147                   341 
##   Nucleus - Chromatin   Nucleus - Nucleolus       Plasma membrane 
##                   143                   322                   326 
##            Proteasome 
##                    21
\end{verbatim}

\begin{itemize}
\tightlist
\item
  \texttt{tagm.map.probability}: the posterior probability for the
  protein sub-cellular allocations.
\end{itemize}

\begin{Shaded}
\begin{Highlighting}[]
\KeywordTok{summary}\NormalTok{(}\KeywordTok{fData}\NormalTok{(E14TG2aR)$tagm.map.probability)}
\end{Highlighting}
\end{Shaded}

\begin{verbatim}
##    Min. 1st Qu.  Median    Mean 3rd Qu.    Max. 
## 0.00000 0.06963 0.93943 0.63829 0.99934 1.00000
\end{verbatim}

\begin{itemize}
\tightlist
\item
  \texttt{tagm.map.outlier}: the posterior probability for that protein
  to belong to the outlier component rather than any annotated
  component.
\end{itemize}

\begin{Shaded}
\begin{Highlighting}[]
\KeywordTok{summary}\NormalTok{(}\KeywordTok{fData}\NormalTok{(E14TG2aR)$tagm.map.outlier)}
\end{Highlighting}
\end{Shaded}

\begin{verbatim}
##      Min.   1st Qu.    Median      Mean   3rd Qu.      Max. 
## 0.0000000 0.0002363 0.0305487 0.3452624 0.9249810 1.0000000
\end{verbatim}

We can visualise the results by scaling the pointer according the
posterior localisation probabilities. To do this we extract the MAP
localisation probabilities from the feature columns of the the
\texttt{MSnSet} and pass these to the \texttt{plot2D} function (figure
\ref{fig:mappca}).

\begin{Shaded}
\begin{Highlighting}[]
\NormalTok{ptsze <-}\StringTok{ }\KeywordTok{fData}\NormalTok{(E14TG2aR)$tagm.map.probability }\CommentTok{# Scale pointer size}
\KeywordTok{plot2D}\NormalTok{(E14TG2aR, }\DataTypeTok{fcol =} \StringTok{"tagm.map.allocation"}\NormalTok{, }\DataTypeTok{cex =} \NormalTok{ptsze)}
\KeywordTok{addLegend}\NormalTok{(E14TG2aR, }\DataTypeTok{where =} \StringTok{"topleft"}\NormalTok{, }\DataTypeTok{cex =} \FloatTok{0.6}\NormalTok{, }\DataTypeTok{fcol =} \StringTok{"tagm.map.allocation"}\NormalTok{)}
\end{Highlighting}
\end{Shaded}

\begin{figure}[htbp]
\centering
\includegraphics{TAGMworkflow_files/figure-latex/mappca-1.pdf}
\caption{\label{fig:mappca}TAGM MAP allocations}
\end{figure}

The TAGM MAP method is easy to use and it is simple to check
convergence, however it is limited in that it can only provide point
estimates of the posterior localisation distributions. To obtain the
full posterior distributions and therefore a rich analysis of the data,
we use Markov-Chain Monte-Carlo methods. In our particular case, we use
a so-called collapsed Gibbs sampler (Smith and Roberts 1993).

\section{\texorpdfstring{Methods: \emph{TAGM MCMC} a brief
overview}{Methods: TAGM MCMC a brief overview}}\label{methods-tagm-mcmc-a-brief-overview}

The TAGM MCMC method allows a fully Bayesian analysis of spatial
proteomics datasets. It employs a collapsed Gibbs sampler to obtain
samples from the posterior distribution of localisation probablities,
providing a rich analysis of the data. This section demonstrates the
advantage of taking a Bayesian approach and the biological information
that can be extracted from this analysis.

Since our audience is unlikely to be versed in Bayesian methodology, we
explain some of the key ideas for a more complete understanding.
Firstly, MCMC based inference constrasts with MAP based inference in
that in produces \textit{samples} from the posterior distribution of
localisation probabilities. Hence, we do not just have a single estimate
for each quantity but a distribution of estimates. MCMC methods are a
large class of algorithms used to sample from a probability
distribution, in our case the posterior distribution of the parameters
(Gilks, Richardson, and Spiegelhalter 1995). They design a Markov-chain;
that is, a random sequence of events where the probability of the next
event only depends on the current state, which, after convergence,
obtains samples from the posterior distribution. A specific example of
an MCMC algorithm is the Gibbs sampler, which can be applied when the
parameters are conditionally conjugate. Often one can perform
Rao-Blackwellisation, a method to reduce posterior variance, to obtain a
collapsed Gibbs sampler (Casella and Robert 1996). Once one has obtained
samples from the posterior distribution, we can estimate the true mean
of the posterior distribution by simply taking the mean of the samples.
In a similar fashion, we can obtain other summaries of the posterior
distribution.

A schematic of MCMC sampling is provided in figure \ref{fig:mcmcCartoon}
to aid understanding. Proteins, coloured blue, are visualised along two
variables of the data. Probability ellipises representing contours of a
probability distribution matching the distribution of the proteins are
overlayed. We now wish to obtain samples from this distribution. The
MCMC algorithm is initialised with a starting location, then at each
iteration a new value is proposed. These proposed values are either
accepted or rejected (according to a carefully computed acceptance
probability) and over many iterations the algorithm converges and this
recipe produces samples from the desired distribution. A large portion
of the earlier samples may not reflect the true distribution, because
the MCMC sampler has yet to converge. These early samples are usually
discarded and this is refered to informally as burnin. The next state of
the algorithm depends on its current state and this leads to
auto-correlation in the samples. To surpress this auto-correlation, we
only retain every \(r^{th}\) sample. This is known as thinning. The
details of burnin and thinning are further detailed in later sections.

\begin{figure}[htbp]
\centering
\includegraphics{TAGMworkflow_files/figure-latex/mcmcCartoon-1.pdf}
\caption{\label{fig:mcmcCartoon}A schematic figure of MCMC sampling}
\end{figure}

The TAGM MCMC method is computationally intensive and requires at least
modest processing power. Leaving the MCMC algorithm to run overnight on
a modern desktop is usually sufficient, however this, of course, depends
on the exact system properties. Do not expect the analysis to finish in
a couple of hours on a medium specification laptop, for example.

To demonstrate the class structure and expected outputs of the TAGM MCMC
method, we run a brief analysis on a subsest (400 randomly chosen
proteins) of the \texttt{tan2009r1} dataset from the
\texttt{pRolocdata}, purely for illustration. This is to provide a bare
bones analysis of these data without being held back by computational
requirements. We perform a complete demonstration and provide precise
details of the analysis of the stem cell dataset considered above in the
next section.

\begin{Shaded}
\begin{Highlighting}[]
\KeywordTok{set.seed}\NormalTok{(}\DecValTok{1}\NormalTok{)}
\KeywordTok{data}\NormalTok{(tan2009r1)}
\NormalTok{tan2009r1 <-}\StringTok{ }\NormalTok{tan2009r1[}\KeywordTok{sample}\NormalTok{(}\KeywordTok{nrow}\NormalTok{(tan2009r1), }\DecValTok{400}\NormalTok{), ]}
\end{Highlighting}
\end{Shaded}

The first step is to run a few MCMC chains (below we use only 2 chains)
for a few iterations (we specify 3 iterations in the below code, but
typically we would suggest in the order of ten thousands, see for
example the algorithms default settings by typing
\texttt{?tagmMcmcTrain}) using the \texttt{tagmMcmcTrain} function. This
function will generate a object of class \texttt{MCMCParams}.

\begin{Shaded}
\begin{Highlighting}[]
\NormalTok{p <-}\StringTok{ }\KeywordTok{tagmMcmcTrain}\NormalTok{(}\DataTypeTok{object =} \NormalTok{tan2009r1, }\DataTypeTok{numIter =} \DecValTok{3}\NormalTok{,}
                   \DataTypeTok{burnin =} \DecValTok{1}\NormalTok{, }\DataTypeTok{thin =} \DecValTok{1}\NormalTok{, }\DataTypeTok{numChains =} \DecValTok{2}\NormalTok{)}
\NormalTok{p}
\end{Highlighting}
\end{Shaded}

\begin{verbatim}
## Object of class "MCMCParams"
## Method: TAGM.MCMC 
## Number of chains: 2
\end{verbatim}

Information for each MCMC chain is contained within the chains slot. If
needed, this information can be accessed manually. The function
\texttt{tagmMcmcProcess} processes the \texttt{MCMCParams} object and
populates the summary slot.

\begin{Shaded}
\begin{Highlighting}[]
\NormalTok{p <-}\StringTok{ }\KeywordTok{tagmMcmcProcess}\NormalTok{(p)}
\NormalTok{p}
\end{Highlighting}
\end{Shaded}

\begin{verbatim}
## Object of class "MCMCParams"
## Method: TAGM.MCMC 
## Number of chains: 2 
## Summary available
\end{verbatim}

The summary slot has now been populated to include basic summaries of
the MCMC chains, such as organelle allocations and localisation
probabilities. Protein information can be appended to the feature
columns of the \texttt{MSnSet} by using the \texttt{tagmPredict}
function, which extracts the required information from the summary slot
of the \texttt{MCMCParams} object.

\begin{Shaded}
\begin{Highlighting}[]
\NormalTok{res <-}\StringTok{ }\KeywordTok{tagmPredict}\NormalTok{(}\DataTypeTok{object =} \NormalTok{tan2009r1, }\DataTypeTok{params =} \NormalTok{p)}
\end{Highlighting}
\end{Shaded}

One can now access new features variables:

\begin{itemize}
\tightlist
\item
  \texttt{tagm.mcmc.allocation}: the TAGM MCMC prediction for the most
  likely protein sub-cellular annotation.
\end{itemize}

\begin{Shaded}
\begin{Highlighting}[]
\KeywordTok{table}\NormalTok{(}\KeywordTok{fData}\NormalTok{(res)$tagm.mcmc.allocation)}
\end{Highlighting}
\end{Shaded}

\begin{verbatim}
## 
##  Cytoskeleton            ER         Golgi      Lysosome mitochondrion 
##            10            98            21            10            41 
##       Nucleus    Peroxisome            PM    Proteasome  Ribosome 40S 
##            25             3           105            29            30 
##  Ribosome 60S 
##            28
\end{verbatim}

\begin{itemize}
\tightlist
\item
  \texttt{tagm.mcmc.probability}: the mean posterior probability for the
  protein sub-cellular allocations.
\end{itemize}

\begin{Shaded}
\begin{Highlighting}[]
\KeywordTok{summary}\NormalTok{(}\KeywordTok{fData}\NormalTok{(res)$tagm.mcmc.probability)}
\end{Highlighting}
\end{Shaded}

\begin{verbatim}
##    Min. 1st Qu.  Median    Mean 3rd Qu.    Max. 
##  0.3096  0.9032  0.9895  0.9093  1.0000  1.0000
\end{verbatim}

As well as other useful summaries of the MCMC methods:

\begin{itemize}
\item
  \texttt{tagm.mcmc.outlier} the posterior probability for the protein
  to belong to the outlier component.
\item
  \texttt{tagm.mcmc.probability.lowerquantile} and
  \texttt{tagm.mcmc.probability.upperquantile} are the lower and upper
  boundaries to the equi-tailed 95\% credible interval of
  \texttt{tagm.mcmc.probability}.
\item
  \texttt{tagm.mcmc.mean.shannon} a Monte-Carlo averaged shannon
  entropy, which is a measure of uncertainty in the allocations.
\end{itemize}

\section{\texorpdfstring{Methods: \emph{TAGM MCMC} the
details}{Methods: TAGM MCMC the details}}\label{methods-tagm-mcmc-the-details}

This section explains how to manually manipulate the MCMC output of the
TAGM model. In the code chunk below, we load a pre-computated TAGM MCMC
model. The data file \texttt{e14tagm.rda} is available online\footnote{\url{https://drive.google.com/open?id=1zozntDhE6YZ-q8wjtQ-lxZ66EEszOGYi}}
and is not directly loaded into this package due to its size. The file
itself if around 500mb, which is too large to directly load into a
package.

\begin{Shaded}
\begin{Highlighting}[]
\KeywordTok{load}\NormalTok{(}\StringTok{"e14Tagm.rda"}\NormalTok{)}
\end{Highlighting}
\end{Shaded}

The following code, which is not evaluated dynamically, was used to
produce the \texttt{tagmE14} \texttt{MCMCParams} object. We run the MCMC
algorithm for 20000 iterations with 10000 iterations discarded for
burnin. We then thin the chain by 20. We ran 6 chains in parallel and so
we obtain 500 samples for each of the 6 chains, totalling 3000 samples.
The resulting file is assumed to be in our working directory.

\begin{Shaded}
\begin{Highlighting}[]
\NormalTok{e14Tagm <-}\StringTok{ }\KeywordTok{tagmMcmcTrain}\NormalTok{(E14TG2aR,}
                         \DataTypeTok{numIter =} \DecValTok{20000}\NormalTok{,}
                         \DataTypeTok{burnin =} \DecValTok{10000}\NormalTok{,}
                         \DataTypeTok{thin =} \DecValTok{20}\NormalTok{,}
                         \DataTypeTok{numChains =} \DecValTok{6}\NormalTok{)}
\end{Highlighting}
\end{Shaded}

Manually inspecting the object, we see that it is a \texttt{MCMCParams}
object with 6 chains.

\begin{Shaded}
\begin{Highlighting}[]
\NormalTok{e14Tagm}
\end{Highlighting}
\end{Shaded}

\begin{verbatim}
## Object of class "MCMCParams"
## Method: TAGM.MCMC 
## Number of chains: 6
\end{verbatim}

\subsection{Data exploration and convergence
diagnostics}\label{data-exploration-and-convergence-diagnostics}

Assessing whether or not an MCMC algorithm has converged is challenging.
Assessing and diagnosing convergence is an active area of research and
throughout the 1990s many approaches were proposed (Geweke 1992; Gelman
and Rubin 1992; Roberts and Smith 1994; Brooks and Gelman 1998). A
converged MCMC algorithm should be oscillating rapidly around a single
value with no monotonicity. We provide a more detailed exploration of
this issue, but the readers should bare in mind that the methods
provided below are diagnostics and cannot guarantee success. We direct
readers to several important works in the literature discussing the
assesment of convergence. Users that do not assess convergence and base
their downstream analysis on unconverged chains are likely to obtain
poor quality results.

We first assess convergence using a parallel chains approach. We find
producing multiple chains is benifical not only for computational
advantages but also for analysis of convergence of our chains.

\begin{Shaded}
\begin{Highlighting}[]
\NormalTok{## Get number of chains}
\NormalTok{nChains <-}\StringTok{ }\KeywordTok{length}\NormalTok{(e14Tagm)}
\NormalTok{nChains}
\end{Highlighting}
\end{Shaded}

\begin{verbatim}
## [1] 6
\end{verbatim}

The following code chunks set up a manual convegence diagnostic check.
We make use of objects and methods in the package
\emph{\href{https://CRAN.R-project.org/package=coda}{coda}} to peform
this analysis (Plummer et al. 2006). Our function below automatically
coerces our objects into
\emph{\href{https://CRAN.R-project.org/package=coda}{coda}} for ease of
analysis. We first calculate the total number of outliers at each
iteration of each chain and, if the algorithm has converged, this number
should be the same (or very similar) across all 6 chains.

\begin{Shaded}
\begin{Highlighting}[]
\NormalTok{## Convergence diagnostic to see if more we need to discard any}
\NormalTok{## iterations or entire chains: compute the number of outliers for}
\NormalTok{## each iteration for each chain}
\NormalTok{out <-}\StringTok{ }\KeywordTok{mcmc_get_outliers}\NormalTok{(e14Tagm)}
\end{Highlighting}
\end{Shaded}

We can observe this from the trace plots and histrograms for each MCMC
chain (figure \ref{fig:mcmctrace1}). Unconverged chains should be
discarded from downstream analysis.

\begin{figure}[htbp]
\centering
\includegraphics{TAGMworkflow_files/figure-latex/mcmctraceHidden-1.pdf}
\caption{\label{fig:mcmctraceHidden}Trace (left) and density (right) of the
6 MCMC chains.}
\end{figure}

\begin{Shaded}
\begin{Highlighting}[]
\NormalTok{## Using coda S3 objects to produce trace plots and histograms}
\NormalTok{for (i in }\KeywordTok{seq_len}\NormalTok{(nChains))}
    \KeywordTok{plot}\NormalTok{(out[[i]], }\DataTypeTok{main =} \KeywordTok{paste}\NormalTok{(}\StringTok{"Chain"}\NormalTok{, i), }\DataTypeTok{auto.layout =} \OtherTok{FALSE}\NormalTok{, }\DataTypeTok{col =} \NormalTok{i)}
\end{Highlighting}
\end{Shaded}

Chains 3, 5 and 6 oscillate around an average of 153, with rapid back
and forth oscillations. Chain 2 should be immediatly discarded, since it
has a large jump in the chain with clearly skewed histogram. The other
two chains oscillate differently with contrasting quantiles to the 3
chains in agreement, suggesting these chains have yet to converge. We
can use the \emph{\href{https://CRAN.R-project.org/package=coda}{coda}}
package to produce summaries of our chains. Here is the \texttt{coda}
summary for the third chain.

\begin{Shaded}
\begin{Highlighting}[]
\NormalTok{## Chains average around 153 outliers}
\KeywordTok{summary}\NormalTok{(out[[}\DecValTok{3}\NormalTok{]])}
\end{Highlighting}
\end{Shaded}

\begin{verbatim}
## 
## Iterations = 1:500
## Thinning interval = 1 
## Number of chains = 1 
## Sample size per chain = 500 
## 
## 1. Empirical mean and standard deviation for each variable,
##    plus standard error of the mean:
## 
##           Mean             SD       Naive SE Time-series SE 
##       153.4520        14.0771         0.6295         0.6820 
## 
## 2. Quantiles for each variable:
## 
##  2.5%   25%   50%   75% 97.5% 
##   127   144   153   162   183
\end{verbatim}

\subsubsection{Applying the Gelman
diagnostic}\label{applying-the-gelman-diagnostic}

Thus far, our analysis appears promising. Three seperate chains
oscillate around an average of 153 outliers and there is no observed
monotonicity in our output. However, for a more rigorous and unbiased
analysis of convergence we can calculate the Gelman diagnostic using the
\emph{\href{https://CRAN.R-project.org/package=coda}{coda}} package
(Gelman and Rubin 1992; Brooks and Gelman 1998). This statistics is
often referred to as \(\hat{R}\) or the potential scale reduction
factor. The idea of the Gelman diagnostics is to compare the inter and
intra chain variances. The ratio of these quantities should be close to
one. The actual statistics computed is more complicated, but we do not
go deeper here and a more detailed and in depth discussion can be found
in the references. The
\emph{\href{https://CRAN.R-project.org/package=coda}{coda}} package also
reports the \(95\%\) upper confidence interval of the \(\hat{R}\)
statistic. In this case, our samples are normally distributed (see
traces on the right in figure \ref{fig:mcmctrace1}). The
\emph{\href{https://CRAN.R-project.org/package=coda}{coda}} package
allows for transformations to improve normality of the data, and in some
cases we set the \texttt{transform} argument to apply log tranformation.
Gelman and Rubin (1992) suggest that chains with \(\hat{R}\) value of
less than 1.2 are likely to have converged.

\begin{Shaded}
\begin{Highlighting}[]
\KeywordTok{gelman.diag}\NormalTok{(out, }\DataTypeTok{transform =} \OtherTok{FALSE}\NormalTok{)}
\end{Highlighting}
\end{Shaded}

\begin{verbatim}
## Potential scale reduction factors:
## 
##      Point est. Upper C.I.
## [1,]       1.14       1.32
\end{verbatim}

\begin{Shaded}
\begin{Highlighting}[]
\KeywordTok{gelman.diag}\NormalTok{(out[}\KeywordTok{c}\NormalTok{(}\DecValTok{1}\NormalTok{,}\DecValTok{3}\NormalTok{,}\DecValTok{4}\NormalTok{,}\DecValTok{5}\NormalTok{,}\DecValTok{6}\NormalTok{)], }\DataTypeTok{transform =} \OtherTok{FALSE}\NormalTok{)}
\end{Highlighting}
\end{Shaded}

\begin{verbatim}
## Potential scale reduction factors:
## 
##      Point est. Upper C.I.
## [1,]       1.13       1.31
\end{verbatim}

\begin{Shaded}
\begin{Highlighting}[]
\KeywordTok{gelman.diag}\NormalTok{(out[}\KeywordTok{c}\NormalTok{(}\DecValTok{3}\NormalTok{,}\DecValTok{5}\NormalTok{,}\DecValTok{6}\NormalTok{)], }\DataTypeTok{transform =} \OtherTok{FALSE}\NormalTok{)}
\end{Highlighting}
\end{Shaded}

\begin{verbatim}
## Potential scale reduction factors:
## 
##      Point est. Upper C.I.
## [1,]          1       1.01
\end{verbatim}

In all cases, we see that the Gelman diagnostic for convergence is
\textless{} 1.2. However, the upper confidence interval is 1.32 when all
chains are used; 1.31 when chain 2 is removed and when chains 1, 2 and 4
are removed the upper confidence interval is 1.01 indicating that the
MCMC algorithm for chains 3,5 and 6 might have converged.

We can also look at the Gelman diagnostics statistics for groups or
pairs of chains. The first line below computes the Gelman diagnostic
across the first three chains, whereas the second calculates between
chain 3 and chain 5.

\begin{Shaded}
\begin{Highlighting}[]
\KeywordTok{gelman.diag}\NormalTok{(out[}\DecValTok{1}\NormalTok{:}\DecValTok{3}\NormalTok{], }\DataTypeTok{transform =} \OtherTok{FALSE}\NormalTok{) }\CommentTok{# the upper C.I is 1.62}
\end{Highlighting}
\end{Shaded}

\begin{verbatim}
## Potential scale reduction factors:
## 
##      Point est. Upper C.I.
## [1,]       1.22       1.62
\end{verbatim}

\begin{Shaded}
\begin{Highlighting}[]
\KeywordTok{gelman.diag}\NormalTok{(out[}\KeywordTok{c}\NormalTok{(}\DecValTok{3}\NormalTok{,}\DecValTok{5}\NormalTok{)], }\DataTypeTok{transform =} \OtherTok{TRUE}\NormalTok{) }\CommentTok{# the upper C.I is 1.01}
\end{Highlighting}
\end{Shaded}

\begin{verbatim}
## Potential scale reduction factors:
## 
##      Point est. Upper C.I.
## [1,]       1.01       1.01
\end{verbatim}

To assess another summary statistic, we can look at the mean component
allocation at each iteration of the MCMC algorithm and as before we
produce trace plots of this quantity (figure \ref{fig:mcmctrace2}).

\begin{Shaded}
\begin{Highlighting}[]
\NormalTok{meanAlloc <-}\StringTok{ }\KeywordTok{mcmc_get_meanComponent}\NormalTok{(e14Tagm)}
\end{Highlighting}
\end{Shaded}

\begin{figure}[htbp]
\centering
\includegraphics{TAGMworkflow_files/figure-latex/mcmctrace2hidden-1.pdf}
\caption{\label{fig:mcmctrace2hidden}Trace (left) and density (right) of the
mean component allocation 6 MCMC chains.}
\end{figure}

\begin{Shaded}
\begin{Highlighting}[]
\NormalTok{for (i in }\KeywordTok{seq_len}\NormalTok{(nChains))}
    \KeywordTok{plot}\NormalTok{(meanAlloc[[i]], }\DataTypeTok{main =} \KeywordTok{paste}\NormalTok{(}\StringTok{"Chain"}\NormalTok{, i), }\DataTypeTok{auto.layout =} \OtherTok{FALSE}\NormalTok{, }\DataTypeTok{col =} \NormalTok{i)}
\end{Highlighting}
\end{Shaded}

As before we can produce summaries of the data.

\begin{Shaded}
\begin{Highlighting}[]
\KeywordTok{summary}\NormalTok{(meanAlloc[[}\DecValTok{1}\NormalTok{]])}
\end{Highlighting}
\end{Shaded}

\begin{verbatim}
## 
## Iterations = 1:500
## Thinning interval = 1 
## Number of chains = 1 
## Sample size per chain = 500 
## 
## 1. Empirical mean and standard deviation for each variable,
##    plus standard error of the mean:
## 
##           Mean             SD       Naive SE Time-series SE 
##       5.686713       0.059112       0.002644       0.002644 
## 
## 2. Quantiles for each variable:
## 
##  2.5%   25%   50%   75% 97.5% 
## 5.552 5.646 5.692 5.728 5.795
\end{verbatim}

We can already observe that there are some slighy difference between
these chains which raises suspicion that some of the chains may not have
converged. For example each chain appears to be oscillating around 5.7,
but chains 2 and 4 have clear bumps in the their trace plots. For a more
quantitaive analysis, we again apply the Gelman diagnostics to these
summaries.

\begin{Shaded}
\begin{Highlighting}[]
\KeywordTok{gelman.diag}\NormalTok{(meanAlloc)}
\end{Highlighting}
\end{Shaded}

\begin{verbatim}
## Potential scale reduction factors:
## 
##      Point est. Upper C.I.
## [1,]          1       1.01
\end{verbatim}

The above values are close to 1 and so we there are no significant
difference between the chains. As observed previously, chains 2 and 4
look quite different from the other chains and so we recalculate the
diagnostic excluding these chains. The computed Gelman diagnostic below
suggest that chains 3, 5 and 6 have converged and that we should discard
chains 1, 2 and 4 from further analysis.

\begin{Shaded}
\begin{Highlighting}[]
\KeywordTok{gelman.diag}\NormalTok{(meanAlloc[}\KeywordTok{c}\NormalTok{(}\DecValTok{3}\NormalTok{,}\DecValTok{5}\NormalTok{,}\DecValTok{6}\NormalTok{)])}
\end{Highlighting}
\end{Shaded}

\begin{verbatim}
## Potential scale reduction factors:
## 
##      Point est. Upper C.I.
## [1,]          1          1
\end{verbatim}

For a further check, we can look at the mean outlier probability at each
iteration of the MCMC algorithm and again computing the Gelman
diagnostics between chains 4, 5 and 6. An \(\hat{R}\) statistics of 1 is
indicative of convergence, since it is less than the recommend value of
1.2.

\begin{Shaded}
\begin{Highlighting}[]
\NormalTok{meanoutProb <-}\StringTok{ }\KeywordTok{mcmc_get_meanoutliersProb}\NormalTok{(e14Tagm)}
\KeywordTok{gelman.diag}\NormalTok{(meanoutProb[}\KeywordTok{c}\NormalTok{(}\DecValTok{3}\NormalTok{, }\DecValTok{5}\NormalTok{, }\DecValTok{6}\NormalTok{)])}
\end{Highlighting}
\end{Shaded}

\begin{verbatim}
## Potential scale reduction factors:
## 
##      Point est. Upper C.I.
## [1,]          1       1.01
\end{verbatim}

\subsubsection{Applying the Geweke
diagnostic}\label{applying-the-geweke-diagnostic}

Along with the Gelman diagnostics, which uses parallel chains, we can
also apply a single chain analysis using the Geweke diagnostic (Geweke
1992). The Geweke diagnostic tests to see whether the mean calculated
from the first \(10\%\) of iterations is significantly different from
the the mean calculated from the last \(50\%\) of iterations. If they
are significantly different, at say a level 0.01, then this is evidence
that particular chains have not converged. The following code chunk
calculates the Geweke diagnostic for each chain on the summarising
quantities we have previously computed.

\begin{Shaded}
\begin{Highlighting}[]
\KeywordTok{geweke_test}\NormalTok{(out)}
\end{Highlighting}
\end{Shaded}

\begin{verbatim}
##           chain 1      chain 2  chain 3    chain 4    chain 5    chain 6
## z.value 0.5749775 8.816632e+00 0.470203 -0.3204500 -0.6270787 -0.7328168
## p.value 0.5653065 1.179541e-18 0.638210  0.7486272  0.5306076  0.4636702
\end{verbatim}

\begin{Shaded}
\begin{Highlighting}[]
\KeywordTok{geweke_test}\NormalTok{(meanAlloc)}
\end{Highlighting}
\end{Shaded}

\begin{verbatim}
##           chain 1       chain 2    chain 3    chain 4   chain 5    chain 6
## z.value 1.1952967 -3.3737051063 -1.2232102 2.48951993 0.3605882 -0.1358850
## p.value 0.2319711  0.0007416377  0.2212503 0.01279157 0.7184073  0.8919122
\end{verbatim}

\begin{Shaded}
\begin{Highlighting}[]
\KeywordTok{geweke_test}\NormalTok{(meanoutProb)}
\end{Highlighting}
\end{Shaded}

\begin{verbatim}
##           chain 1      chain 2   chain 3    chain 4    chain 5     chain 6
## z.value 0.1785882 1.205500e+01 0.6189637 -0.5164987 -0.2141086 -0.02379004
## p.value 0.8582611 1.825379e-33 0.5359403  0.6055062  0.8304624  0.98102008
\end{verbatim}

The first test suggest chain 2 has not converged, since the p-value is
less than \(10^{-10}\) suggesting that the mean in the first \(10\%\) of
iterations is significantly different from those in the final \(50\%\).
Moreover, the second test and third tests also suggest chain 2 has not
converged. Furthermore, for the second test chain 4 has a marginally
small p-value and thus further evidence that this chain is of low
quality. These convergence diagnostics are not limited to the quantities
we have computed here and further diagnostics can be perform on any
summary of the data.

An important question to consider is whether removing an early portion
of the chain might lead to improvment of the convergence diagonistics.
This might be particularly relevant if a chain converges some iterations
after our orginally specified \texttt{burin}. For example let us take
the second Geweke test above, which suggested chains 2 and 4 had not
converged and see if discarding the initial \(10\%\) of the chain
improves the statistic. The function below removes \(50\) samples ,
informaly known as \texttt{burnin}, from the beginning of each chain and
the output shows that we now have \(450\) samples in each chain. In
practice, as \(2\) chains are sufficient for good posterior estimates
and convergence we could simply discard chains \(2\) and \(4\) and
proceed with downstream analysis with the remaining \(4\) chains.

\begin{Shaded}
\begin{Highlighting}[]
\NormalTok{burne14Tagm <-}\StringTok{ }\KeywordTok{mcmc_burn_chains}\NormalTok{(e14Tagm, }\DecValTok{50}\NormalTok{)}
\KeywordTok{chains}\NormalTok{(burne14Tagm)}
\end{Highlighting}
\end{Shaded}

\begin{verbatim}
## Object of class "MCMCChains"
##  Number of chains: 6
\end{verbatim}

\begin{Shaded}
\begin{Highlighting}[]
\KeywordTok{chains}\NormalTok{(burne14Tagm)[[}\DecValTok{4}\NormalTok{]]}
\end{Highlighting}
\end{Shaded}

\begin{verbatim}
## Object of class "MCMCChain"
##  Number of components: 10 
##  Number of proteins: 1663 
##  Number of iterations: 450
\end{verbatim}

The following function recomputes the number of outliers in each chain
at each iteration of each Markov-chain.

\begin{Shaded}
\begin{Highlighting}[]
\NormalTok{newout <-}\StringTok{ }\KeywordTok{mcmc_get_outliers}\NormalTok{(burne14Tagm)}
\end{Highlighting}
\end{Shaded}

The code chuck below computes the Geweke diagonstic for this new
truncated chain and demonstrates that chain 4 has an improved Geweke
diagnostic, whilst chain 2 does not. Thus, in practice, it maybe useful
to remove iterations from the beginning of the chain. However, as chain
4 did not pass the Gelman diagnostics we still discard it from
downstream analysis.

\begin{Shaded}
\begin{Highlighting}[]
\KeywordTok{geweke_test}\NormalTok{(newout)}
\end{Highlighting}
\end{Shaded}

\begin{verbatim}
##            chain 1      chain 2    chain 3   chain 4   chain 5   chain 6
## z.value -0.1455345 6.379618e+00 -1.6392215 0.3836940 0.1241201 0.6654703
## p.value  0.8842889 1.775298e-10  0.1011671 0.7012053 0.9012202 0.5057497
\end{verbatim}

\subsection{Processing converged
chains}\label{processing-converged-chains}

Having made an assessment of convergence, we decide to discard chains
\(1,2\) and \(4\) from any further analysis. The code chunk below remove
these chains and creates and new object to store the converged chains.

\begin{Shaded}
\begin{Highlighting}[]
\NormalTok{removeChain <-}\StringTok{ }\KeywordTok{c}\NormalTok{(}\DecValTok{1}\NormalTok{, }\DecValTok{2}\NormalTok{, }\DecValTok{4}\NormalTok{) }\CommentTok{# The chains to be removed}
\NormalTok{e14Tagm_converged <-}\StringTok{ }\NormalTok{e14Tagm[-removeChain] }\CommentTok{# Create new object}
\end{Highlighting}
\end{Shaded}

The \texttt{MCMCParams} object can be large and therefore if we have a
large number of samples we may want to subsample our chain, informally
known as thinning, to reduce the number of samples. Thinning also has
another purpose. We may desire indepedent samples from our posterior
distribution but the MCMC algorithm produces autocorrelated samples.
Thinning can be applied to reduce the autocorrelation between samples.
The code chuck below, which is not evaluated, demonstrates retaining
every \(5^{th}\) iteration. Recall that we thinned by \(20\) when we
first ran the MCMC algorithm.

\begin{Shaded}
\begin{Highlighting}[]
\NormalTok{e14Tagm_converged_thinned <-}\StringTok{ }\KeywordTok{mcmc_thin_chains}\NormalTok{(e14Tagm_converged, }\DataTypeTok{freq  =} \DecValTok{5}\NormalTok{)}
\end{Highlighting}
\end{Shaded}

We initially ran \(6\) chains and, after having made an assesssment of
convergence, we decided to discard \(3\) of the chains. We desire to
make inference using samples from all \(3\) chains, since this leads to
better posterior estimates. In their current class structure all the
chains are stored separately, so the following function pools all sample
for all chains together to make a single longer chain with all samplers.
Pooling a mixture of converged and unconverged chains is likely to lead
to poor quality results so should be done with care.

\begin{Shaded}
\begin{Highlighting}[]
\NormalTok{e14Tagm_converged_pooled <-}\StringTok{ }\KeywordTok{mcmc_pool_chains}\NormalTok{(e14Tagm_converged)}
\NormalTok{e14Tagm_converged_pooled}
\end{Highlighting}
\end{Shaded}

\begin{verbatim}
## Object of class "MCMCParams"
## Method: TAGM.MCMC 
## Number of chains: 1
\end{verbatim}

\begin{Shaded}
\begin{Highlighting}[]
\NormalTok{e14Tagm_converged_pooled[[}\DecValTok{1}\NormalTok{]]}
\end{Highlighting}
\end{Shaded}

\begin{verbatim}
## Object of class "MCMCChain"
##  Number of components: 10 
##  Number of proteins: 1663 
##  Number of iterations: 1500
\end{verbatim}

To populate the summary slot of the converged and pooled chain, we can
use the \texttt{tagmMcmcProcess} function. As we can see from the object
below a summary is now available. The information now available in the
summary slot was detailed in the previous section. We note that if there
is more than \(1\) chain in the \texttt{MCMCParams} object then the
chains are automatically pooled to compute the summaries.

\begin{Shaded}
\begin{Highlighting}[]
\NormalTok{e14Tagm_converged_pooled <-}\StringTok{ }\KeywordTok{tagmMcmcProcess}\NormalTok{(e14Tagm_converged_pooled)}
\NormalTok{e14Tagm_converged_pooled}
\end{Highlighting}
\end{Shaded}

\begin{verbatim}
## Object of class "MCMCParams"
## Method: TAGM.MCMC 
## Number of chains: 1 
## Summary available
\end{verbatim}

To create new feature columns in the \texttt{MSnSet} and appened the
summary information, we apply the \texttt{tagmPredict} function. The
\texttt{probJoint} argument indicates whether or not to add probablistic
information for all organelles for all proteins, rather than just the
information for the most probable organelle. The outlier probabilities
are also return by default, but users can change this using the
\texttt{probOutlier} argument.

\begin{Shaded}
\begin{Highlighting}[]
\NormalTok{E14TG2aR <-}\StringTok{ }\KeywordTok{tagmPredict}\NormalTok{(}\DataTypeTok{object =} \NormalTok{E14TG2aR,}
                        \DataTypeTok{params =} \NormalTok{e14Tagm_converged_pooled,}
                        \DataTypeTok{probJoint =} \OtherTok{TRUE}\NormalTok{)}
\KeywordTok{head}\NormalTok{(}\KeywordTok{fData}\NormalTok{(E14TG2aR))}
\end{Highlighting}
\end{Shaded}

\begin{verbatim}
##        Uniprot.ID UniprotName
## Q62261     Q62261 SPTB2_MOUSE
## Q9JHU4     Q9JHU4 DYHC1_MOUSE
## Q9QXS1     Q9QXS1  PLEC_MOUSE
## P16546     P16546 SPTA2_MOUSE
## Q69ZN7     Q69ZN7  MYOF_MOUSE
## P30999     P30999 CTND1_MOUSE
##                                     Protein.Description Peptides PSMs
## Q62261 Spectrin beta chain, brain 1 (multiple isoforms)       42   42
## Q9JHU4               Cytoplasmic dynein 1 heavy chain 1       33   33
## Q9QXS1                       Isoform PLEC-1I of Plectin       33   33
## P16546  Spectrin alpha chain, brain (multiple isoforms)       32   32
## Q69ZN7                    Myoferlin (multiple isoforms)       28   28
## P30999              Catenin delta-1 (multiple isoforms)       24   24
##        GOannotation markers.orig         markers   tagm.map.allocation
## Q62261      PLM-SKE      unknown         unknown Endoplasmic reticulum
## Q9JHU4          SKE      unknown         unknown   Nucleus - Chromatin
## Q9QXS1      unknown      unknown         unknown       Plasma membrane
## P16546  PLM-SKE-CYT      unknown         unknown   Nucleus - Chromatin
## Q69ZN7          VES      unknown         unknown       Plasma membrane
## P30999      PLM-NUC          PLM Plasma membrane       Plasma membrane
##        tagm.map.probability tagm.map.outlier  tagm.mcmc.allocation
## Q62261         8.165817e-09     0.9999999857 Endoplasmic reticulum
## Q9JHU4         9.996798e-01     0.0003202255   Nucleus - Chromatin
## Q9QXS1         1.250898e-06     0.9999987491            Proteasome
## P16546         4.226696e-07     0.9999995462 Endoplasmic reticulum
## Q69ZN7         9.994502e-01     0.0001083130       Plasma membrane
## P30999         1.000000e+00     0.0000000000       Plasma membrane
##        tagm.mcmc.probability tagm.mcmc.probability.lowerquantile
## Q62261             0.5765793                        0.0020296117
## Q9JHU4             0.9738206                        0.7594516090
## Q9QXS1             0.4957129                        0.0002886457
## P16546             0.5214374                        0.0014041362
## Q69ZN7             0.9997025                        0.9981794326
## P30999             1.0000000                        1.0000000000
##        tagm.mcmc.probability.upperquantile tagm.mcmc.mean.shannon
## Q62261                           0.9992504            0.201623229
## Q9JHU4                           0.9998822            0.081450206
## Q9QXS1                           0.9947100            0.447665536
## P16546                           0.9946959            0.252833750
## Q69ZN7                           0.9999954            0.002395147
## P30999                           1.0000000            0.000000000
##        tagm.mcmc.outlier tagm.mcmc.joint.40S Ribosome
## Q62261      2.547793e-01                 4.401228e-10
## Q9JHU4      3.335134e-05                 1.936225e-18
## Q9QXS1      6.423799e-01                 2.213861e-07
## P16546      2.119112e-01                 1.576023e-09
## Q69ZN7      7.274103e-06                 3.510523e-22
## P30999      0.000000e+00                 0.000000e+00
##        tagm.mcmc.joint.60S Ribosome tagm.mcmc.joint.Cytosol
## Q62261                 2.778620e-07            2.650861e-12
## Q9JHU4                 1.645727e-21            1.887645e-17
## Q9QXS1                 1.495170e-01            9.062280e-09
## P16546                 3.150122e-06            1.471329e-08
## Q69ZN7                 5.152312e-16            2.063009e-24
## P30999                 0.000000e+00            0.000000e+00
##        tagm.mcmc.joint.Endoplasmic reticulum tagm.mcmc.joint.Lysosome
## Q62261                          5.765793e-01             1.108757e-11
## Q9JHU4                          1.548053e-17             5.577415e-24
## Q9QXS1                          1.768681e-04             1.150706e-04
## P16546                          5.214374e-01             3.687975e-09
## Q69ZN7                          8.397027e-09             2.974966e-04
## P30999                          0.000000e+00             0.000000e+00
##        tagm.mcmc.joint.Mitochondrion tagm.mcmc.joint.Nucleus - Chromatin
## Q62261                  5.020528e-08                        4.231731e-01
## Q9JHU4                  2.835919e-22                        9.738206e-01
## Q9QXS1                  5.832273e-19                        7.920397e-03
## P16546                  4.522032e-08                        4.776913e-01
## Q69ZN7                  6.143974e-39                        4.872032e-21
## P30999                  0.000000e+00                        0.000000e+00
##        tagm.mcmc.joint.Nucleus - Nucleolus tagm.mcmc.joint.Plasma membrane
## Q62261                        1.279255e-05                    1.914808e-11
## Q9JHU4                        2.617943e-02                    3.514851e-29
## Q9QXS1                        1.130580e-05                    3.465462e-01
## P16546                        3.448558e-05                    2.489652e-07
## Q69ZN7                        7.042891e-30                    9.997025e-01
## P30999                        0.000000e+00                    1.000000e+00
##        tagm.mcmc.joint.Proteasome
## Q62261               2.345204e-04
## Q9JHU4               7.841425e-11
## Q9QXS1               4.957129e-01
## P16546               8.333595e-04
## Q69ZN7               1.003778e-10
## P30999               0.000000e+00
\end{verbatim}

\subsubsection*{\texorpdfstring{Aside:
\emph{Priors}}{Aside: Priors}}\label{aside-priors}
\addcontentsline{toc}{subsubsection}{Aside: \emph{Priors}}

Bayesian analysis requires users to specify prior information about the
parameters. This potentially appears a challenging task; however, good
default options are often possible. Should expert information be
available for any of these priors then the users should provide this,
else we have found that the default choices work well in practice. The
priors also pro vide regularisation and shrinkage to avoid overfitting.
Given enough data the likelihood overwhelms the prior and little
inference is based on the prior. There is little guidance in the
literature on how to choose priors or even what they mean for the data
analysis.

We place a normal inverse-Wishart prior on the normally distributed
mixture components. The normal inverse-Wishart prior has \(4\)
hyperparameters to be chosen to specify the prior beliefs in the mean
and covariance of these mixtures. These are the prior mean \texttt{mu0}
expressing the prior location of each organelle; a prior shrinkage
\texttt{lambda0} a scaler expressing uncertainty in the prior mean; the
prior degress of freedom \texttt{nu0} and a scale prior \texttt{S0} on
the covariance. Together \texttt{nu0} and \texttt{S0} specify the prior
variability on organelle covariances. All components share the same
prior information.

The default options for these hold little information and are based on
choice recommended by (Fraley and Raftery 2005). The prior mean
\texttt{mu0} is set to be the mean of the data. \texttt{lambda0} is set
to be \(0.01\) meaning some uncertainty in the covariance is propograted
to the mean, increasing \texttt{lambda0} increases shrinkage towards the
prior. \texttt{nu0} is set to the number of feature variables plus
\(2\), which is the smallest integer value that ensures a finite
covariance matrix. The prior scale matrix \(S0\) is set to

\begin{equation}
S_0 = \frac{\diag(\frac{1}{n}\sum (X - \bar{X})^2)}{K^{1/D}},
\end{equation}

and represents a diffuse prior on the covariance. Another good choice
which is often used is a constant multiple of the identity matrix. The
prior for the Dirichlet distribution concentration paramters
\texttt{beta0} is set to \(1\) for each organelle. This represents a
symmetric belief about the number of proteins allocated to each
organelle. Another reasonable choice would be the non-informative
Jeffery's prior for the Dirichlet hyperparameter, which sets
\texttt{beta0} to \(0.5\) for each organelle. The prior weight for the
outlier detection class is a \(\mathcal{B}(u, v)\) distribution. The
default for \(u = 2\) and the default for \(v = 10\). This represents
the reasonable belief that \(\frac{u}{u + v} = \frac{1}{6}\) proteins
\emph{a priori} might be an outlier and we believe is unlikely that more
than \(50\%\) of proteins are outliers. Decreasing the value of \(v\),
represents more uncertainty about the number of protein that are
outliers.

\subsection{Analysis, visualisation and interpretation of
results}\label{analysis-visualisation-and-interpretation-of-results}

Now that we have single pooled chain of samples from a converged MCMC
algorithm, we can begin to analyse the results. Preliminary analysis
includes visualising the allocated organelle and localisation probabilty
of each protein to its most probable organelle, as shown on figure
\ref{fig:mcmcpca}.

\begin{Shaded}
\begin{Highlighting}[]
\KeywordTok{par}\NormalTok{(}\DataTypeTok{mfrow =} \KeywordTok{c}\NormalTok{(}\DecValTok{1}\NormalTok{, }\DecValTok{2}\NormalTok{))}
\KeywordTok{plot2D}\NormalTok{(E14TG2aR, }\DataTypeTok{fcol =} \StringTok{"tagm.mcmc.allocation"}\NormalTok{,}
       \DataTypeTok{cex =} \KeywordTok{fData}\NormalTok{(E14TG2aR)$tagm.mcmc.probability,}
       \DataTypeTok{main =} \StringTok{"TAGM MCMC allocations"}\NormalTok{)}
\KeywordTok{addLegend}\NormalTok{(E14TG2aR, }\DataTypeTok{fcol =} \StringTok{"markers"}\NormalTok{,}
          \DataTypeTok{where =} \StringTok{"topleft"}\NormalTok{, }\DataTypeTok{ncol =} \DecValTok{2}\NormalTok{, }\DataTypeTok{cex =} \FloatTok{0.6}\NormalTok{)}

\KeywordTok{plot2D}\NormalTok{(E14TG2aR, }\DataTypeTok{fcol =} \StringTok{"tagm.mcmc.allocation"}\NormalTok{,}
       \DataTypeTok{cex =} \KeywordTok{fData}\NormalTok{(E14TG2aR)$tagm.mcmc.mean.shannon,}
       \DataTypeTok{main =} \StringTok{"Visualising global uncertainty"}\NormalTok{)}
\KeywordTok{addLegend}\NormalTok{(E14TG2aR, }\DataTypeTok{fcol =} \StringTok{"markers"}\NormalTok{,}
          \DataTypeTok{where =} \StringTok{"topleft"}\NormalTok{, }\DataTypeTok{ncol =} \DecValTok{2}\NormalTok{, }\DataTypeTok{cex =} \FloatTok{0.6}\NormalTok{)}
\end{Highlighting}
\end{Shaded}

\begin{figure}[htbp]
\centering
\includegraphics{TAGMworkflow_files/figure-latex/mcmcpca-1.pdf}
\caption{\label{fig:mcmcpca}TAGM MCMC allocations. On the left, point size
have been scaled based on allocation probabilities. On the right, the
point size have been scaled based on the global uncertainty using the
mean Shannon entropy.}
\end{figure}

We can visualise other summaries of the data including a Monte-Carlo
averaged Shannon entropy, as show on figure \ref{fig:mcmcpca} on the
right. This is a measure of uncertainty and proteins with greater
shannon entropy have more uncertainty in their localisation. We observe
global patterns of uncertainty, particulaly in areas where organelle
boundaries overlap. There are also regions of low uncertainty indicating
little doubt about the localisation of these proteins.

We are also interested in the relatonship between localisation
probability to the most probable class and the Shannon entropy.
Eventhough the two quantities are evidently correlated there is still
considerable spread. Thus it is important to base inference not only on
localisation probability but also a measure of uncertainty, for example
the Shannon entropy. Proteins with low Shannon entropy have low
uncertainty in their localisation, whlist those with higher Shannon
entropy have uncertain localisation. Since multi-localised protein have
uncertain localisation to a single sub-celullar niche, exploring the
Shannon can aid in identifying multi-localised proteins.

\begin{Shaded}
\begin{Highlighting}[]
\NormalTok{cls <-}\StringTok{ }\KeywordTok{getStockcol}\NormalTok{()[}\KeywordTok{as.factor}\NormalTok{(}\KeywordTok{fData}\NormalTok{(E14TG2aR)$tagm.mcmc.allocation)]}
\KeywordTok{plot}\NormalTok{(}\KeywordTok{fData}\NormalTok{(E14TG2aR)$tagm.mcmc.probability,}
     \KeywordTok{fData}\NormalTok{(E14TG2aR)$tagm.mcmc.mean.shannon,}
     \DataTypeTok{col =} \NormalTok{cls, }\DataTypeTok{pch =} \DecValTok{19}\NormalTok{,}
     \DataTypeTok{xlab =} \StringTok{"Localisation probability"}\NormalTok{,}
     \DataTypeTok{ylab =} \StringTok{"Shannon entropy"}\NormalTok{)}
\KeywordTok{addLegend}\NormalTok{(E14TG2aR, }\DataTypeTok{fcol =} \StringTok{"markers"}\NormalTok{,}
          \DataTypeTok{where =} \StringTok{"topright"}\NormalTok{, }\DataTypeTok{ncol =} \DecValTok{2}\NormalTok{, }\DataTypeTok{cex =} \FloatTok{0.6}\NormalTok{)}
\end{Highlighting}
\end{Shaded}

\begin{figure}[htbp]
\centering
\includegraphics{TAGMworkflow_files/figure-latex/probvsentr-1.pdf}
\caption{\label{fig:probvsentr}Shannon entropy and localisation
probability.}
\end{figure}

Aside from global visualisation of the data, we can also interrogate
each individual protein. As illustrated on figure \ref{fig:probdists1},
we can obtain the full posterior distribution of localisation
probabilities for each protein from the
\texttt{e14Tagm\_converged\_pooled} object. We can use the \texttt{plot}
generic on the \texttt{MCMCParams} object to obtain a violin plot of the
localisation distribution. Simply providing the name of the protein in
the second argument produces the plot for that protein. The solute
carrier transporter protein E9QMX3, also refered to as Slc15a1, is most
probably localised to plasma membrane in line with its role as a
transmembrane transporter but also shows some uncertainty, potentially
also localising to other comparments. The first violin plot visualises
this uncertainty. The protein Q3V1Z5 is a supposed constitute of the 40S
ribosome and has poor UniProt annotation with evidence only at the
transcript level. From the plot below is is clear that Q3V1Z5 is a
ribosomal associated protein, but it previous localisation has only been
computational inferred and here we provide experimental evidence of a
ribosomal annotation. Thus, quantifying uncertainty recovers important
additional annotations.

\begin{Shaded}
\begin{Highlighting}[]
\KeywordTok{plot}\NormalTok{(e14Tagm_converged_pooled, }\StringTok{"E9QMX3"}\NormalTok{)}
\KeywordTok{plot}\NormalTok{(e14Tagm_converged_pooled, }\StringTok{"Q3V1Z5"}\NormalTok{)}
\end{Highlighting}
\end{Shaded}

\begin{figure}[htbp]
\centering
\includegraphics{TAGMworkflow_files/figure-latex/probdists1-1.pdf}
\caption{\label{fig:probdists1}Full posterior distribution of localisation
probabilities for individual proteins.}
\end{figure}

\section{Discussion}\label{discussion}

The Bayesian analysis of biological data is of clear interest to many
because of its ability to provide richer information about the
experimental results. A fully Bayesian analysis differs from other
machine learning approaches, since it can quantify the uncertainty in
our experiments. Furthermore, a generative model is used to explicitly
describe the data rendering the output more interpretable, rather than
the less intepretable outputs of black-box classifiers such as, for
example, support vector machines (SVM).

Bayesian analysis if often characterised by its provision of a
probability distribution over the biological parameters of interest, as
opposed to single point estimate of these parameters. In the case that
is presented in this workflow, a Bayesian analysis ``computes'' a
posterior probability distribution over the protein localisation
probabilities. These probability distributions can then be rigorously
interrogated for greater biological insight; in addition, it may allow
us to ask additional questions about the data, such as whether a protein
might be multi-localised.

Despite the wealth of information a Bayesian analysis can provide, the
uptake amongst cell biologists is still low. This is because a Bayesian
analysis provides a new set of challenges with sparse guidance on it
practical implementation. Bayesian analysis often relies on complex
algorithms such as Markov-chain Monte-Carlo (MCMC) and a practical
understanding of these algorithms and the interpretation of the output
is a key barrier to their use. A Bayesian analysis usually consistes of
three broad sections: (1) Data pre-processing and algorithmic
implementation, (2) assessing algorithmic convergence and (3)
summarising and visualising the results. This workflow provides a set of
tools to simplify these steps and provides step by step guidance in the
context of the analysis of spatial proteomics data.

We have provided a workflow for the Bayesian analysis of spatial
proteomics using the \texttt{pRoloc} and \texttt{MSnbase} software. We
have demonstrated, in a step by step fashion, the challenges and
advantages to taking a Bayesian approach to data analysis. We hope this
workflow helps spatial proteomics practitioners to apply our methods and
inspire others to create detailed documentation for the Bayesian
analysis of biolgical data.

\section{Session information}\label{session-information}

Below, we provide a summary of all packages and versions used to
generate this document.

\begin{Shaded}
\begin{Highlighting}[]
\KeywordTok{sessionInfo}\NormalTok{()}
\end{Highlighting}
\end{Shaded}

\begin{verbatim}
## R version 3.5.2 (2018-12-20)
## Platform: x86_64-w64-mingw32/x64 (64-bit)
## Running under: Windows 10 x64 (build 17134)
## 
## Matrix products: default
## 
## locale:
## [1] LC_COLLATE=English_United Kingdom.1252 
## [2] LC_CTYPE=English_United Kingdom.1252   
## [3] LC_MONETARY=English_United Kingdom.1252
## [4] LC_NUMERIC=C                           
## [5] LC_TIME=English_United Kingdom.1252    
## 
## attached base packages:
## [1] stats4    parallel  stats     graphics  grDevices utils     datasets 
## [8] methods   base     
## 
## other attached packages:
##  [1] patchwork_0.0.1      pRolocdata_1.20.0    pRoloc_1.23.2       
##  [4] coda_0.19-2          mixtools_1.1.0       BiocParallel_1.16.5 
##  [7] MLInterfaces_1.62.0  cluster_2.0.7-1      annotate_1.60.0     
## [10] XML_3.98-1.17        AnnotationDbi_1.44.0 IRanges_2.16.0      
## [13] MSnbase_2.8.3        ProtGenerics_1.14.0  S4Vectors_0.20.1    
## [16] mzR_2.16.1           Rcpp_1.0.0           Biobase_2.42.0      
## [19] BiocGenerics_0.28.0 
## 
## loaded via a namespace (and not attached):
##   [1] snow_0.4-3            plyr_1.8.4            igraph_1.2.4         
##   [4] lazyeval_0.2.1        splines_3.5.2         ggvis_0.4.4          
##   [7] crosstalk_1.0.0       ggplot2_3.1.0         digest_0.6.18        
##  [10] foreach_1.4.4         htmltools_0.3.6       viridis_0.5.1        
##  [13] gdata_2.18.0          magrittr_1.5          memoise_1.1.0        
##  [16] doParallel_1.0.14     sfsmisc_1.1-3         limma_3.38.3         
##  [19] recipes_0.1.4         gower_0.1.2           rda_1.0.2-2.1        
##  [22] lpSolve_5.6.13        prettyunits_1.0.2     colorspace_1.4-0     
##  [25] blob_1.1.1            xfun_0.5              dplyr_0.8.0.1        
##  [28] crayon_1.3.4          RCurl_1.95-4.11       hexbin_1.27.2        
##  [31] genefilter_1.64.0     impute_1.56.0         survival_2.43-3      
##  [34] iterators_1.0.10      glue_1.3.0            gtable_0.2.0         
##  [37] ipred_0.9-8           zlibbioc_1.28.0       kernlab_0.9-27       
##  [40] prabclus_2.2-7        DEoptimR_1.0-8        scales_1.0.0         
##  [43] vsn_3.50.0            mvtnorm_1.0-8         DBI_1.0.0            
##  [46] viridisLite_0.3.0     xtable_1.8-3          progress_1.2.0       
##  [49] bit_1.1-14            proxy_0.4-22          mclust_5.4.2         
##  [52] preprocessCore_1.44.0 lava_1.6.5            prodlim_2018.04.18   
##  [55] sampling_2.8          htmlwidgets_1.3       httr_1.4.0           
##  [58] threejs_0.3.1         FNN_1.1.3             RColorBrewer_1.1-2   
##  [61] fpc_2.1-11.1          modeltools_0.2-22     pkgconfig_2.0.2      
##  [64] flexmix_2.3-15        nnet_7.3-12           caret_6.0-81         
##  [67] labeling_0.3          tidyselect_0.2.5      rlang_0.3.1          
##  [70] reshape2_1.4.3        later_0.8.0           munsell_0.5.0        
##  [73] mlbench_2.1-1         tools_3.5.2           LaplacesDemon_16.1.1 
##  [76] generics_0.0.2        RSQLite_2.1.1         pls_2.7-0            
##  [79] evaluate_0.13         stringr_1.4.0         mzID_1.20.1          
##  [82] yaml_2.2.0            ModelMetrics_1.2.2    knitr_1.21           
##  [85] bit64_0.9-7           robustbase_0.93-3     randomForest_4.6-14  
##  [88] purrr_0.3.0           dendextend_1.9.0      ncdf4_1.16           
##  [91] nlme_3.1-137          whisker_0.3-2         mime_0.6             
##  [94] biomaRt_2.38.0        compiler_3.5.2        rstudioapi_0.9.0     
##  [97] e1071_1.7-0.1         affyio_1.52.0         tibble_2.0.1         
## [100] stringi_1.3.1         highr_0.7             lattice_0.20-38      
## [103] trimcluster_0.1-2.1   Matrix_1.2-15         gbm_2.1.5            
## [106] pillar_1.3.1          BiocManager_1.30.4    MALDIquant_1.18      
## [109] data.table_1.12.0     bitops_1.0-6          httpuv_1.4.5.1       
## [112] R6_2.4.0              pcaMethods_1.74.0     affy_1.60.0          
## [115] hwriter_1.3.2         bookdown_0.9          promises_1.0.1       
## [118] gridExtra_2.3         codetools_0.2-15      MASS_7.3-51.1        
## [121] gtools_3.8.1          assertthat_0.2.0      withr_2.1.2          
## [124] diptest_0.75-7        hms_0.4.2             timeDate_3043.102    
## [127] grid_3.5.2            rpart_4.1-13          class_7.3-14         
## [130] rmarkdown_1.11        segmented_0.5-3.0     lubridate_1.7.4      
## [133] shiny_1.2.0           base64enc_0.1-3
\end{verbatim}

The source of this document, including the code necessary to reproduce
the analyses and figures is available in a public manuscript repository
on GitHub (Crook and Gatto 2019).

\section*{Reference}\label{reference}
\addcontentsline{toc}{section}{Reference}

\hypertarget{refs}{}
\hypertarget{ref-Beltran:2016}{}
Beltran, Pierre M Jean, Rommel A Mathias, and Ileana M Cristea. 2016.
``A Portrait of the Human Organelle Proteome in Space and Time During
Cytomegalovirus Infection.'' \emph{Cell Systems} 3 (4). Elsevier:
361--73.

\hypertarget{ref-Breckels:2013}{}
Breckels, Lisa M, Laurent Gatto, Andy Christoforou, Arnoud J Groen,
Kathryn S Lilley, and Matthew WB Trotter. 2013. ``The Effect of
Organelle Discovery Upon Sub-Cellular Protein Localisation.''
\emph{Journal of Proteomics} 88. Elsevier: 129--40.

\hypertarget{ref-Breckels:2016}{}
Breckels, Lisa M, Sean B Holden, David Wojnar, Claire M Mulvey, Andy
Christoforou, Arnoud Groen, Matthew WB Trotter, Oliver Kohlbacher,
Kathryn S Lilley, and Laurent Gatto. 2016. ``Learning from Heterogeneous
Data Sources: An Application in Spatial Proteomics.'' \emph{PLoS
Computational Biology} 12 (5). Public Library of Science: e1004920.

\hypertarget{ref-Breckels:2016b}{}
Breckels, Lisa M, Claire M Mulvey, Kathryn S Lilley, and Laurent Gatto.
2016. ``A Bioconductor Workflow for Processing and Analysing Spatial
Proteomics Data.'' \emph{F1000Research} 5.

\hypertarget{ref-Brooks:1998}{}
Brooks, Stephen P, and Andrew Gelman. 1998. ``General Methods for
Monitoring Convergence of Iterative Simulations.'' \emph{Journal of
Computational and Graphical Statistics} 7 (4). Taylor \& Francis:
434--55.

\hypertarget{ref-Casella:1996}{}
Casella, George, and Christian P Robert. 1996. ``Rao-Blackwellisation of
Sampling Schemes.'' \emph{Biometrika} 83 (1). Oxford University Press:
81--94.

\hypertarget{ref-hyper}{}
Christoforou, Andy, Claire M Mulvey, Lisa M Breckels, Aikaterini
Geladaki, Tracey Hurrell, Penelope C Hayward, Thomas Naake, et al. 2016.
``A Draft Map of the Mouse Pluripotent Stem Cell Spatial Proteome.''
\emph{Nature Communications} 7. Nature Publishing Group: 9992.

\hypertarget{ref-Cody:2013}{}
Cody, Neal AL, Carole Iampietro, and Eric Lécuyer. 2013. ``The Many
Functions of MRNA Localization During Normal Development and Disease:
From Pillar to Post.'' \emph{Wiley Interdisciplinary Reviews:
Developmental Biology} 2 (6). Wiley Online Library: 781--96.

\hypertarget{ref-Crook:2018}{}
Crook, Oliver M, Claire M Mulvey, Paul D W Kirk, Kathryn S Lilley, and
Laurent Gatto. 2018. ``A Bayesian Mixture Modelling Approach for Spatial
Proteomics.'' \emph{PLoS Comput. Biol.} 14 (11): e1006516.

\hypertarget{ref-ghrepo}{}
Crook, OM, and L Gatto. 2019. ``A Bioconductor Workflow for the Bayesian
Analysis of Spatial Proteomics.'' \emph{GitHub Repository}.
\url{https://github.com/ococrook/TAGMworkflow}; GitHub.

\hypertarget{ref-Davies:2018}{}
Davies, Alexandra K, Daniel N Itzhak, James R Edgar, Tara L Archuleta,
Jennifer Hirst, Lauren P Jackson, Margaret S Robinson, and Georg HH
Borner. 2018. ``AP-4 Vesicles Contribute to Spatial Control of Autophagy
via Rusc-Dependent Peripheral Delivery of Atg9a.'' \emph{Nature
Communications} 9. Nature Publishing Group: 3958.

\hypertarget{ref-De:2011}{}
De Matteis, Maria Antonietta, and Alberto Luini. 2011. ``Mendelian
Disorders of Membrane Trafficking.'' \emph{New England Journal of
Medicine} 365 (10). Mass Medical Soc: 927--38.

\hypertarget{ref-EM:1977}{}
Dempster, Arthur P, Nan M Laird, and Donald B Rubin. 1977. ``Maximum
Likelihood from Incomplete Data via the Em Algorithm.'' \emph{Journal of
the Royal Statistical Society. Series B (Methodological)}. JSTOR, 1--38.

\hypertarget{ref-Dunkley:2006}{}
Dunkley, Tom PJ, Svenja Hester, Ian P Shadforth, John Runions, Thilo
Weimar, Sally L Hanton, Julian L Griffin, et al. 2006. ``Mapping the
Arabidopsis Organelle Proteome.'' \emph{Proceedings of the National
Academy of Sciences} 103 (17). National Acad Sciences: 6518--23.

\hypertarget{ref-Dunkley:2004}{}
Dunkley, Tom PJ, Rod Watson, Julian L Griffin, Paul Dupree, and Kathryn
S Lilley. 2004. ``Localization of Organelle Proteins by Isotope Tagging
(Lopit).'' \emph{Molecular \& Cellular Proteomics} 3 (11). ASBMB:
1128--34.

\hypertarget{ref-Foster:2006}{}
Foster, Leonard J, Carmen L de Hoog, Yanling Zhang, Yong Zhang, Xiaohui
Xie, Vamsi K Mootha, and Matthias Mann. 2006. ``A Mammalian Organelle
Map by Protein Correlation Profiling.'' \emph{Cell} 125 (1). Elsevier:
187--99.

\hypertarget{ref-Fraley:2005}{}
Fraley, Chris, and Adrian E Raftery. 2005. ``Bayesian Regularization for
Normal Mixture Estimation and Model-Based Clustering.'' \emph{Techincal
Report}. Washington Univ Seattle Dept of Statistics.

\hypertarget{ref-pRoloc:2014}{}
Gatto, Laurent, Lisa M. Breckels, Samuel Wieczorek, Thomas Burger, and
Kathryn S. Lilley. 2014. ``Mass-Spectrometry Based Spatial Proteomics
Data Analysis Using PRoloc and PRolocdata.'' \emph{Bioinformatics}.

\hypertarget{ref-DC:2018}{}
Geladaki, Aikaterini, Nina Kocevar Britovsek, Lisa M Breckels, Tom Sand
Owen L Vennard Smith, Claire M Mulvey, Oliver M Crook, Laurent Gatto,
and Kathryn S Lilley. 2019. ``Combining Lopit with Differential
Ultracentrifugation for High-Resolution Spatial Proteomics.''
\emph{Nature Communications} 10. Nature Publishing Group: 331.

\hypertarget{ref-Gelman:1995}{}
Gelman, A., J. B. Carlin, H. S. Stern, and D. B. Rubin. 1995.
\emph{Bayesian Data Analysis}. London: Chapman \& Hall.

\hypertarget{ref-Gelman:1992}{}
Gelman, Andrew, and Donald B Rubin. 1992. ``Inference from Iterative
Simulation Using Multiple Sequences.'' \emph{Statistical Science}.
JSTOR, 457--72.

\hypertarget{ref-Geweke:1992}{}
Geweke, John. 1992. ``Evaluating the Accuracy of Sampling-Based
Approaches to the Calculation of Posterior Moments.'' \emph{BAYESIAN
STATISTICS}. Citeseer.

\hypertarget{ref-Gibson:2009}{}
Gibson, Toby J. 2009. ``Cell Regulation: Determined to Signal Discrete
Cooperation.'' \emph{Trends in Biochemical Sciences} 34 (10). Elsevier:
471--82.

\hypertarget{ref-Gilks:1995}{}
Gilks, Walter R, Sylvia Richardson, and David Spiegelhalter. 1995.
\emph{Markov Chain Monte Carlo in Practice}. Chapman; Hall/CRC.

\hypertarget{ref-Hall:2009}{}
Hall, Stephanie L, Svenja Hester, Julian L Griffin, Kathryn S Lilley,
and Antony P Jackson. 2009. ``The Organelle Proteome of the Dt40
Lymphocyte Cell Line.'' \emph{Molecular \& Cellular Proteomics} 8 (6).
ASBMB: 1295--1305.

\hypertarget{ref-Hirst:2018}{}
Hirst, Jennifer, Daniel N Itzhak, Robin Antrobus, Georg HH Borner, and
Margaret S Robinson. 2018. ``Role of the Ap-5 Adaptor Protein Complex in
Late Endosome-to-Golgi Retrieval.'' \emph{PLoS Biology} 16 (1). Public
Library of Science: e2004411.

\hypertarget{ref-Itzhak:2017}{}
Itzhak, Daniel N, Colin Davies, Stefka Tyanova, Archana Mishra, James
Williamson, Robin Antrobus, Jürgen Cox, Michael P Weekes, and Georg HH
Borner. 2017. ``A Mass Spectrometry-Based Approach for Mapping Protein
Subcellular Localization Reveals the Spatial Proteome of Mouse Primary
Neurons.'' \emph{Cell Reports} 20 (11). Elsevier: 2706--18.

\hypertarget{ref-Itzhak:2016}{}
Itzhak, Daniel N, Stefka Tyanova, Jürgen Cox, and Georg HH Borner. 2016.
``Global, Quantitative and Dynamic Mapping of Protein Subcellular
Localization.'' \emph{Elife} 5. eLife Sciences Publications Limited:
e16950.

\hypertarget{ref-Jadot:2017}{}
Jadot, Michel, Marielle Boonen, Jaqueline Thirion, Nan Wang, Jinchuan
Xing, Caifeng Zhao, Abla Tannous, et al. 2017. ``Accounting for Protein
Subcellular Localization: A Compartmental Map of the Rat Liver
Proteome.'' \emph{Molecular \& Cellular Proteomics} 16 (2). ASBMB:
194--212.

\hypertarget{ref-Jeffery:2009}{}
Jeffery, Constance J. 2009. ``Moonlighting Proteins - an Update.''
\emph{Molecular BioSystems} 5 (4). Royal Society of Chemistry: 345--50.

\hypertarget{ref-Kau:2004}{}
Kau, Tweeny R, Jeffrey C Way, and Pamela A Silver. 2004. ``Nuclear
Transport and Cancer: From Mechanism to Intervention.'' \emph{Nature
Reviews Cancer} 4 (2). Nature Publishing Group: 106--17.

\hypertarget{ref-Latorre:2005}{}
Latorre, Isabel J, Michael H Roh, Kristopher K Frese, Robert S Weiss,
Ben Margolis, and Ronald T Javier. 2005. ``Viral Oncoprotein-Induced
Mislocalization of Select Pdz Proteins Disrupts Tight Junctions and
Causes Polarity Defects in Epithelial Cells.'' \emph{Journal of Cell
Science} 118 (18). The Company of Biologists Ltd: 4283--93.

\hypertarget{ref-Laurila:2009}{}
Laurila, Kirsti, and Mauno Vihinen. 2009. ``Prediction of
Disease-Related Mutations Affecting Protein Localization.'' \emph{BMC
Genomics} 10 (1). BioMed Central: 122.

\hypertarget{ref-Luheshi:2008}{}
Luheshi, Leila M, Damian C Crowther, and Christopher M Dobson. 2008.
``Protein Misfolding and Disease: From the Test Tube to the Organism.''
\emph{Current Opinion in Chemical Biology} 12 (1). Elsevier: 25--31.

\hypertarget{ref-Mendes:2017}{}
Mendes, Marta, Alberto Peláez-García, María López-Lucendo, Rubén A.
Bartolomé, Eva Calviño, Rodrigo Barderas, and J. Ignacio Casal. 2017.
``Mapping the Spatial Proteome of Metastatic Cells in Colorectal
Cancer.'' \emph{Proteomics} 17 (19).
doi:\href{https://doi.org/10.1002/pmic.201700094}{10.1002/pmic.201700094}.

\hypertarget{ref-Mulvey:2017}{}
Mulvey, Claire M, Lisa M Breckels, Aikaterini Geladaki, Nina Kočevar
Britovšek, Daniel JH Nightingale, Andy Christoforou, Mohamed Elzek,
Michael J Deery, Laurent Gatto, and Kathryn S Lilley. 2017. ``Using
hyperLOPIT to Perform High-Resolution Mapping of the Spatial Proteome.''
\emph{Nature Protocols} 12 (6). Nature Research: 1110--35.

\hypertarget{ref-Nightingale:2019}{}
Nightingale, Daniel JH, Aikaterini Geladaki, Lisa M Breckels, Stephen G
Oliver, and Kathryn S Lilley. 2019. ``The Subcellular Organisation of
Saccharomyces Cerevisiae.'' \emph{Current Opinion in Chemical Biology}
48. Elsevier: 86--95.

\hypertarget{ref-Olkkonen:2006}{}
Olkkonen, Vesa M, and Elina Ikonen. 2006. ``When Intracellular Logistics
Fails-Genetic Defects in Membrane Trafficking.'' \emph{Journal of Cell
Science} 119 (24). The Company of Biologists Ltd: 5031--45.

\hypertarget{ref-Orre:2019}{}
Orre, Lukas Minus, Mattias Vesterlund, Yanbo Pan, Taner Arslan, Yafeng
Zhu, Alejandro Fernandez Woodbridge, Oliver Frings, Erik Fredlund, and
Janne Lehtiö. 2019. ``SubCellBarCode: Proteome-Wide Mapping of Protein
Localization and Relocalization.'' \emph{Molecular Cell} 73 (1):
166--182.e7.
doi:\href{https://doi.org/https://doi.org/10.1016/j.molcel.2018.11.035}{https://doi.org/10.1016/j.molcel.2018.11.035}.

\hypertarget{ref-coda}{}
Plummer, Martyn, Nicky Best, Kate Cowles, and Karen Vines. 2006. ``CODA:
Convergence Diagnosis and Output Analysis for Mcmc.'' \emph{R News} 6
(1): 7--11. \url{https://journal.r-project.org/archive/}.

\hypertarget{ref-Roberts:1994}{}
Roberts, Gareth O, and Adrian FM Smith. 1994. ``Simple Conditions for
the Convergence of the Gibbs Sampler and Metropolis-Hastings
Algorithms.'' \emph{Stochastic Processes and Their Applications} 49 (2).
Elsevier: 207--16.

\hypertarget{ref-Rodriguez:2004}{}
Rodriguez, José Antonio, Wendy WY Au, and Beric R Henderson. 2004.
``Cytoplasmic Mislocalization of Brca1 Caused by Cancer-Associated
Mutations in the Brct Domain.'' \emph{Experimental Cell Research} 293
(1). Elsevier: 14--21.

\hypertarget{ref-Shin:2013}{}
Shin, Soo J, Jeffrey A Smith, Günther A Rezniczek, Sheng Pan, Ru Chen,
Teresa A Brentnall, Gerhard Wiche, and Kimberly A Kelly. 2013.
``Unexpected Gain of Function for the Scaffolding Protein Plectin Due to
Mislocalization in Pancreatic Cancer.'' \emph{Proceedings of the
National Academy of Sciences} 110 (48). National Acad Sciences:
19414--9.

\hypertarget{ref-Siljee:2018}{}
Siljee, J E, Y Wang, A A Bernard, B A Ersoy, S Zhang, A Marley, M Von
Zastrow, J F Reiter, and C Vaisse. 2018. ``Subcellular Localization of
MC4R with ADCY3 at Neuronal Primary Cilia Underlies a Common Pathway for
Genetic Predisposition to Obesity.'' \emph{Nat Genet}, January.
doi:\href{https://doi.org/10.1038/s41588-017-0020-9}{10.1038/s41588-017-0020-9}.

\hypertarget{ref-Smith:1993}{}
Smith, Adrian FM, and Gareth O Roberts. 1993. ``Bayesian Computation via
the Gibbs Sampler and Related Markov Chain Monte Carlo Methods.''
\emph{Journal of the Royal Statistical Society. Series B
(Methodological)}. JSTOR, 3--23.

\hypertarget{ref-Tan:2009}{}
Tan, Denise JL, Heidi Dvinge, Andrew Christoforou, Paul Bertone, Alfonso
Martinez Arias, and Kathryn S Lilley. 2009. ``Mapping Organelle Proteins
and Protein Complexes in Drosophila Melanogaster.'' \emph{Journal of
Proteome Research} 8 (6). ACS Publications: 2667--78.

\hypertarget{ref-Thul:2017}{}
Thul, Peter J, Lovisa Åkesson, Mikaela Wiking, Diana Mahdessian,
Aikaterini Geladaki, Hammou Ait Blal, Tove Alm, et al. 2017. ``A
Subcellular Map of the Human Proteome.'' \emph{Science} 356 (6340).
American Association for the Advancement of Science: eaal3321.


\end{document}
